\section{Navigazione Incognita}
\begin{frame}{Navigazione Incognita}
	\centering
	\includegraphics[width=0.50\textwidth]{navinc}
\end{frame}
\begin{frame}{Navigazione Incognita}
\begin{alertblock}{Che cosa è?}{Con la navigazione privata, il browser crea una sessione temporanea isolata dalla sessione principale e dai dati dell'utente. La cronologia di navigazione non viene salvata e i dati locali associati alla sessione, come i cookie, vengono cancellati alla chiusura della sessione stessa. Le varie implementazioni sono progettate principalmente per impedire che dati e cronologia associati a una particolare sessione di navigazione persistano sul dispositivo o vengano scoperti da un altro utente operante sullo stesso dispositivo.\footnote{Navigazione privata, url = "\url{//it.wikipedia.org/w/index.php?title=Navigazione_privata&oldid=139810917}", (in data 30 luglio 2024).}}
\end{alertblock}
\end{frame}
\begin{frame}{Navigazione Incognita}
	\begin{alertblock}{Stai navigando in incognito}{
			Le altre persone che usano questo dispositivo non vedranno la tua attività, di conseguenza potrai navigare in modo più privato. La modalità di raccolta dei dati da parte dei siti web visitati e dei servizi utilizzati, incluso Google, rimane invariata. I download, i preferiti e gli elementi dell'elenco di lettura verranno salvati. }
	\end{alertblock}
\end{frame}
\begin{frame}{Navigazione Incognita}
	\begin{alertblock}{Stai navigando in incognito}{
			Chrome non salverà quanto segue:
			\begin{itemize}
				\item Cronologia di navigazione
				\item Cookie e dati dei siti
				\item Informazioni inserite nei moduli
			\end{itemize}}
	\end{alertblock}
		\begin{alertblock}{Stai navigando in incognito}{
			La tua attività potrebbe comunque essere visibile:
			\begin{itemize}
				\item Ai siti web visitati
				\item Al tuo datore di lavoro o alla tua scuola
				\item Al tuo provider di servizi Internet
			\end{itemize}}
	\end{alertblock}
\end{frame}
\begin{frame}{Navigazione Incognita}
	\begin{alertblock}{Ma allora, a che cosa serve?}{Con la navigazione privata, il browser crea una sessione temporanea isolata dalla sessione principale e dai dati dell'utente.\footnote{Navigazione privata, url = "\url{//it.wikipedia.org/w/index.php?title=Navigazione_privata&oldid=139810917}", (in data 30 luglio 2024).}}
	\end{alertblock}
\end{frame}