\section{Gruppi}
\subsection{Cosa è un gruppo?}
\begin{frame}{Cosa è un gruppo?}
	\begin{alertblock}{Definizione}
	Un gruppo è un insieme di utenti o altro che hanno lo stesso interesse.
	\end{alertblock}
	Un gruppo ha:
	\begin{itemize}
		\item Un proprietario
		\item Una mail
		\item Privacy
		\item Utenti
	\end{itemize}
\end{frame}
\begin{frame}{A cosa serve un gruppo?}
	\begin{block}{}
		Comunicare tramite la mail del gruppo
	\end{block}
	\begin{block}{}
		Configurare
	\end{block}
\end{frame}
\subsection{Nomi per i gruppi}
\begin{frame}{Regole per i nomi}
	\begin{alertblock}{Nomi}
		Ogni Classe è formata da \textbf{A}lunni, ha i suoi \textbf{D}ocenti, ha un suo \textbf{C}oordinatore, ha un nome, ed esiste in un anno scolastico.
	\end{alertblock}
	\begin{block}{Esempi}
		\begin{itemize}
			\item 	Il gruppo alunni della 1A sarà \textbf{A1a12425}
			\item 	Il gruppo insegnanti della 1A sarà \textbf{D1a12425}
			\item 	Il Coordinatore  1A avrà \textbf{C1a12425}
		\end{itemize}
	\end{block}
\end{frame}