\section{Sotto il cofano}
\subsection{Perché dobbiamo cancellare gli utenti?}
\begin{frame}{Perché dobbiamo cancellare gli utenti?}
	\begin{itemize}
		\item<1-> Garante della Privacy
		\item<2-> Obblighi contrattuali
		\item<3-> Numero di licenze disponibili
	\end{itemize}
\end{frame}
\begin{frame}{Garante della Privacy}
	\begin{exampleblock}{Garante della Privacy}
		Abbiamo in deposito documenti personali e quindi, alla cessazione, tali documenti devono essere distrutti.
	\end{exampleblock}
\end{frame}
\begin{frame}{Obblighi contrattuali}
	\begin{block}{Obblighi contrattuali}
		La scuola può fornire questi servizi solo a studenti e personale scolastico che appartengono a questa scuole.
	\end{block}
\end{frame}
\begin{frame}{Numero di licenze disponibili}
	\begin{alertblock}{Numero di licenze disponibili}
		Il problema è Canva che ci ha fornito solo di un numero limitato di licenze.
	\end{alertblock}
\end{frame}
\subsection{Proprietario}
\begin{frame}[standout]{Proprietario}
	\begin{block}{Proprietario}
		Qualunque utente che crei qualcosa in WorkSpace è un proprietario
	\end{block}
\end{frame}
\begin{frame}{Proprietario e gli altri}
	\begin{block}{Proprietario e gli altri}
		Un Proprietario concede agli altri le sue cose impostando vari livelli di accesso.
	\end{block}
	\begin{alertblock}{Accesso}
		Con i livelli di accesso non viene trasferita la proprietà  
	\end{alertblock}
	\begin{exampleblock}{Esistenza}
		Il documento, drive, etc. esiste finché  l'utente esiste.
	\end{exampleblock}
\end{frame}
\subsection{Drive personali Vs Drive condivisi}
\begin{frame}{Drive personali Vs Drive condivisi}
	\begin{alertblock}{Drive personali}
		Condivido con gli altri ma mantengo la proprietà
	\end{alertblock}
	\begin{block}{Drive Condivisi}
		Condivido con gli altri ma perdo la proprietà
	\end{block}
\end{frame}