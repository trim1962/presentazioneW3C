\documentclass{beamer}
\usetheme{moloch}
\useinnertheme{moloch}
\usefonttheme{moloch}
\setbeamertemplate{itemize items}[triangle]
%\usecolortheme{seahorse} 
\title{Worspace 365 Canva}
\author{\texorpdfstring{John Doe\newline\url{info@iisperugia.net}}{John Doe}}
%\author{}
\begin{document}
\begin{frame}%[plain]
    \maketitle
 
 
\end{frame}
\begin{frame}{Sommario}
	\tableofcontents
\end{frame}
\section{Introduzione}
\begin{frame}{Quattro chiacchiere}
	La nostra scuola ha 3 ambienti 
	\begin{itemize}[(I)]
		\item<1-> Google Workspace for Education Fundamentals
		\item<2-> Microsoft 365
		\item<3-> Canva
	\end{itemize}
\end{frame}
\section{Google Workspace}
\begin{frame}{Google Workspace}
 Google Workspace ha tre compiti
	\begin{itemize}[(I)]
		\item<1-> Gestire la posta
		\item<2-> Gsiute 
		\item<3-> SSO
		\begin{itemize}
			\item<4->Microsof 365
			\item<5->Canva
		\end{itemize}
	\end{itemize}
\end{frame}
\section{SSO}
\begin{frame}{SSO}
	\begin{alertblock}{SSO}
	Single Sign-On (SSO) è uno strumento di autenticazione che consente agli utenti di accedere in modo sicuro a più applicazioni e servizi utilizzando un unico set di credenziali, eliminando la necessità di ricordare password diverse per ciascun servizio\footnote{Informazione ricavata da https://www.okta.com/blog/2021/02/single-sign-on-sso/ consultato il 27/07/2024}.
	\end{alertblock}
\end{frame}
\begin{frame}{Unico utente e unica password}
	\begin{alertblock}{Unico utente e unica password}
	nome.cognome.doc@iisperugia.net
	
	password 
	
Danno accesso a tutte le piattaforme della scuola
	\end{alertblock}
\end{frame}
\begin{frame}[ standout ]
\centering
\Huge\bfseries
\textcolor{orange}{The End}
\end{frame}
\end{document}
